\documentclass[]{article}
\usepackage{lmodern}
\usepackage{amssymb,amsmath}
\usepackage{ifxetex,ifluatex}
\usepackage{fixltx2e} % provides \textsubscript
\ifnum 0\ifxetex 1\fi\ifluatex 1\fi=0 % if pdftex
  \usepackage[T1]{fontenc}
  \usepackage[utf8]{inputenc}
\else % if luatex or xelatex
  \ifxetex
    \usepackage{mathspec}
  \else
    \usepackage{fontspec}
  \fi
  \defaultfontfeatures{Ligatures=TeX,Scale=MatchLowercase}
\fi
% use upquote if available, for straight quotes in verbatim environments
\IfFileExists{upquote.sty}{\usepackage{upquote}}{}
% use microtype if available
\IfFileExists{microtype.sty}{%
\usepackage{microtype}
\UseMicrotypeSet[protrusion]{basicmath} % disable protrusion for tt fonts
}{}
\usepackage[margin=1in]{geometry}
\usepackage{hyperref}
\hypersetup{unicode=true,
            pdftitle={Modelo Matematico CBR},
            pdfauthor={P.N. de Lima, D. B. Goldmeyer, L. F. R. Camargo, A. Dresch, D. P. Lacerda, T. Kunrath},
            pdfborder={0 0 0},
            breaklinks=true}
\urlstyle{same}  % don't use monospace font for urls
\usepackage{graphicx,grffile}
\makeatletter
\def\maxwidth{\ifdim\Gin@nat@width>\linewidth\linewidth\else\Gin@nat@width\fi}
\def\maxheight{\ifdim\Gin@nat@height>\textheight\textheight\else\Gin@nat@height\fi}
\makeatother
% Scale images if necessary, so that they will not overflow the page
% margins by default, and it is still possible to overwrite the defaults
% using explicit options in \includegraphics[width, height, ...]{}
\setkeys{Gin}{width=\maxwidth,height=\maxheight,keepaspectratio}
\IfFileExists{parskip.sty}{%
\usepackage{parskip}
}{% else
\setlength{\parindent}{0pt}
\setlength{\parskip}{6pt plus 2pt minus 1pt}
}
\setlength{\emergencystretch}{3em}  % prevent overfull lines
\providecommand{\tightlist}{%
  \setlength{\itemsep}{0pt}\setlength{\parskip}{0pt}}
\setcounter{secnumdepth}{5}
% Redefines (sub)paragraphs to behave more like sections
\ifx\paragraph\undefined\else
\let\oldparagraph\paragraph
\renewcommand{\paragraph}[1]{\oldparagraph{#1}\mbox{}}
\fi
\ifx\subparagraph\undefined\else
\let\oldsubparagraph\subparagraph
\renewcommand{\subparagraph}[1]{\oldsubparagraph{#1}\mbox{}}
\fi

%%% Use protect on footnotes to avoid problems with footnotes in titles
\let\rmarkdownfootnote\footnote%
\def\footnote{\protect\rmarkdownfootnote}

%%% Change title format to be more compact
\usepackage{titling}

% Create subtitle command for use in maketitle
\newcommand{\subtitle}[1]{
  \posttitle{
    \begin{center}\large#1\end{center}
    }
}

\setlength{\droptitle}{-2em}
  \title{Modelo Matematico CBR}
  \pretitle{\vspace{\droptitle}\centering\huge}
  \posttitle{\par}
  \author{P.N. de Lima, D. B. Goldmeyer, L. F. R. Camargo, A. Dresch, D. P.
Lacerda, T. Kunrath}
  \preauthor{\centering\large\emph}
  \postauthor{\par}
  \predate{\centering\large\emph}
  \postdate{\par}
  \date{junho de 2017}


\begin{document}
\maketitle

{
\setcounter{tocdepth}{6}
\tableofcontents
}
\section{Modelo Matemático - Razão
Benefício-Custo}\label{modelo-matematico---razao-beneficio-custo}

Este documento contém uma definição do modelo matemático que suporta a
calculadora de custos e benefícios de inciativas em SST.

\subsection{CBR - Razão
Benefício-Custo}\label{cbr---razao-beneficio-custo}

A razão benefício-custo \(\alpha\) corresponde à razão do somatório dos
custos \(C_i\) onde \(i\) representa o índice de custos e \(B_j\) os
benefícios a valor presente.

\[\alpha = \frac{\sum_{i=1}^{I} B_{i}} {\sum_{j=1}^{J} C_{j}}\]

\subsubsection{Fluxo de Caixa em Valor
Presente}\label{fluxo-de-caixa-em-valor-presente}

Os fluxos de caixa devem ser ajustados a valor presente utilizando-se
uma taxa de atratividade \(\theta\) definida pelo usuário do modelo. Tal
taxa será utilizada para trazer os valores de fluxo de caixa a valor
presente.

\[B_i(t) = \frac{b_i}{(1+\theta)^t}\]

\subsubsection{Calculo dos Benefícios}\label{calculo-dos-beneficios}

Em todos os casos, o benefício será calculado a partir da diferença em
valores monetários de uma variável financeira sem a iniciativa em SST e
com a iniciativa em SST. Exemplificando, o benefício gerado pela redução
de absenteísmo \(B_{abs}\) será calculado a partir da seguinte equação.

\[B_i = {D}_{i, ci} - {D}_{i, si}\] Exemplificando, se uma empresa, sem
uma iniciativa em SST terá \(20000\) reais em desepesas com absenteísmo,
e com esta iniciativa terá \(15000\), o benefício oriúndo desta
inciativa, apenas relacionado a absenteísmo será:

\[B_{abs} = {D}_{abs, ci} - {D}_{abs, si} = (-15000)-(-20000) = 5000\]

\paragraph{Despesas Evitáveis}\label{despesas-evitaveis}

\subparagraph{Despesas com Reclamatórias
Trabalhistas}\label{despesas-com-reclamatorias-trabalhistas}

Esta subcategoria compreende as despesas evitadas com reclamatórias
trabalhistas (objeto da ação relacionadas à doenças e acidentes do
trabalho) após a implementação integral da iniciativa.

Parâmetros: \(c_{med}\): Custo Médio da Reclamatória
\(n_{reclamatorias}\): Número de Reclamatórias trabalhistas relacionadas
a SST.

\[{D}_{reclamatorias} = c_{med}*n_{reclamatorias} \]

Número de Reclamatórias Trabalhistas

\[n_{reclamatorias} = f_{desligados} * p_{ajuizarEganharreclamatoria} \]

Parâmetros: \(p_{ajuizarEganharreclamatoria}\): Probabilidade de um
funcionário demitido entrar com uma reclamatória e ganhar a causa.
\(f_{desligados}\): Número de Funcionários desligados pela empresa.

Possibilidade 1: Considerar uma única probabilidade de ajuizar e ganhar
uma ação reclamatória. Esta probabilidade vezes o número de funcionários
gera o número de reclamatórias trabalhistas a pagar.

Possibilidade 2: Separar o ajuizamento da ação do ganho da ação. Desta
maneira é possível calcular o custo em ações mesmo quando o funcionário
não ganhou a ação. A princípio estamos na possibilidade 1.

\subparagraph{Acidente / Doença Ocupacional -
Invalidez}\label{acidente-doenca-ocupacional---invalidez}

Esta subcategoria compreende as despesas evitadas com incapacitação
parcial ou total provocada por acidente típico, doença ocupacional ou
acidente de trajeto após a implementação integral da iniciativa.

Possibilidade 1: Todos os custos incorridos nesta rúbrica entram para o
calculo do FAP e não deveriam ser contados em duplicidade. Possibilidade
2: Existem despesas que não estão em nenhuma outra categoria e que
deveriam ser contabilizados aqui. A princípio estamos na possibilidade
1. A categoria será excluída caso a possibilidade 1 se confirme.

\subparagraph{Ações Regressivas (NTEP)}\label{acoes-regressivas-ntep}

Esta subcategoria compreende as despesas evitadas com ações regressivas
do INSS após a implementação integral da iniciativa.

A Ação Regressiva representa o o ressarcimento de pagamento de
benefícios acidentários do empregador ao INSS. Lei 8213/91, artigo 120
:A ação regressiva é a penalização adicional relacionada ao B91 - B94.

Ações Regressivas Relacionadas ao INSS

\[D_{ações regressivas INSS} =  \sum_{i=1}^{B} n_iacumulado * p_iacaoregressiva * (f_{crise}*crise) * tregressiva_{i}  \]

Dúvida: O NTEP já estará contabilizado integralmente pelo FAP? Sim:
Então precisa estar aqui. Não: Então Dúvida 2: Crise econômica deve
modular esta probabilidade? Que variáveis de contexto devem modular que
variáveis de input?

Ações Regressivas Relacionadas ao Plano de Saúde

\[D_{ações regressivas SUS} =  TotaldeAfastamentosAcumulado * PrecentualTratamentoNoSUS * (f_{crise}*crise) * PercentualCobrancaSUS/PlanodeSaude * ticketmedio\]

\subparagraph{Ausência para Tratamento}\label{ausencia-para-tratamento}

Esta subcategoria compreende as despesas evitadas com a ausência do
trabalhador afastado para tratamento após a implementação integral da
iniciativa.

Esta categoria irá para a categoria de absenteísmo.

\subparagraph{Despesas Médicas}\label{despesas-medicas}

Esta subcategoria compreende as despesas evitadas com medicamento e
atendimento médico para tratamento dos acidentes de trabalho após a
implementação integral da iniciativa.

\[D_{medicas} = (\sum_{k=1}^{K} n_{acidentesk}) * d_{medio}\]

\subparagraph{Redução de Valores do plano de
Saúde}\label{reducao-de-valores-do-plano-de-saude}

Esta subcategoria compreende as despesas evitadas com planos de saúde
via alteração da taxa de sinistralidade após a implementação integral da
iniciativa. \[D_{planosaude} = D_{planosaudebase} * desc_{plano}\]

Dúvida: Como calcular o desconto no plano de saúde a partir das demais
variáveis? \[desc_{plano}(TaxaSinistralidade) =  ?? \]

\subparagraph{Reabilitação:}\label{reabilitacao}

Usará (f\_\{crise\}*crise)

\subparagraph{Interrupção Operacional por
Acidente/Morte}\label{interrupcao-operacional-por-acidentemorte}

\[D_{interdicao} = n_{acidentestipico} * dias * lucrocessante\]

\subparagraph{Interdições Por
Fiscalização}\label{interdicoes-por-fiscalizacao}

\[D_{interdicao} = p_{interdicao} * (f_{crise}*crise) * dias * lucrocessante \]

\subparagraph{Reabilitação do
trabalhador}\label{reabilitacao-do-trabalhador}

\paragraph{Benefício Não Capturado}\label{beneficio-nao-capturado}

\subparagraph{Exposição à Multas}\label{exposicao-a-multas}

\subparagraph{Reduções Fiscais}\label{reducoes-fiscais}

FAP, RAT e SAT: As despesas com seguro acidentário do trabalho
\(D_{sat}\) serão calculadas de acordo com as estimativas do \(FAP\)
(\(0,005 - 0,02\)) e \(RAT\) .

\[D_{sat} = RAT_{ajust}* F\]

Rat Ajustado

\[RAT_{ajust} = (FAP * RAT)\] O \(RAT\) varia entre 1 e 3, de acordo com
o cnae da empresa em questão. \[RAT \in {1,2,3}\]

FAP

O FAP, por sua vez, é calculado de acordo com os percentis de gravidade
\(p_g\), frequência \(p_{f}\) e custo\(p_c\):

\[FAP = (0,5*p_g + 0,35*p_{f}+0,15*p_c)0,02\]

Percentis são calculados de acordo com os índices nos dois anos
anteriores.* Os percentis dependem do posicionamento da empresa em
relação às demais. Específicamente a função \(Pos(I_{t-1},I_{t-2})\) é
calculada pela previdência de acordo com os índices de todas as empresas
no mesmo subgrupo do CNAE da empresa em questão. Ainda não foi definida
uma maneira de estimar esta função.

\[p_t = \frac{100*(Pos(I_{t-1},I_{t-2})-1)}{n-1}\]

Será definida uma forma para estimar \(p_t\).

Índice de Frequência

\[I_f = \frac{(n_{cats}+n_{b92}+n_{b91}+n_{b93})}{f} * 1000\]

Índice de gravidade

\[I_g = \frac{(0.1*n_{b91}+0.3*n_{b92}+0.5*n_{b93}+0.1*n_{b94})}{f}* 1000\]

Índice de Custo

\[I_c = \frac{c_{beneficios inss}}{folha media}/ * 1000\]

Custos de Benefícios: Considerar tempo médio de afastamento por tipo de
afastamento (b91, b92, b93, b94) e o ``ticket médio'' de cada um destes
benefícios.

\subparagraph{B91, B92, B93, B,94, etc}\label{b91-b92-b93-b94-etc}

\[b_i = n_i * p_i \ \forall b_i \in B\]

Obs: Será necessário possuir

\paragraph{Intangível}\label{intangivel}

\subparagraph{Imagem da Empresa}\label{imagem-da-empresa}

\subparagraph{Engajamento / Clima
Organizacional}\label{engajamento-clima-organizacional}

\paragraph{Melhor Uso dos Recursos}\label{melhor-uso-dos-recursos}

\subparagraph{Turnover}\label{turnover}

As despesas com Turnover \(D_{tur}\) serão calculadas com base no número
de funcionários afastados por problemas relacionados à SST \(n_{afast}\)
e no custo médio de substituição dos funcionários\(c_{sub}\).

\[D_{tur} = n_{afast} * c_{sub}\]

Número de Afastamentos é calculado de acordo com a probabilidade de
morte \(p_{morte}\) e a probabilidade de afastamento por período menor
que 15 dias \(p_{>15}\)

\[n_{afast} = p_{morte}*f + p_{>15}*f\]

Número de Afastamentos

O número de afastamentos \(n_{afast}\) será estimado de acordo com a
probabilidade de afastamento \(\rho_{afast}\) e número de funcionários
\(f\): \[n_{afast} = \rho_{afast} * f \]

\subparagraph{Absenteísmo}\label{absenteismo}

As despesas com Absenteísmo \(D_{abs}\) serão calculadas com base no
número de dias de absenteísmo por problemas relacionados à SST
\(d_{abs}\), no número de horas trabalhadas por dia \(h\) e no custo em
mão de obra médio horário \(c_{mdo}\).

\[D_{abs} = d_{abs} * h * c_{mdo}\]

Dias de Absenteísmo

\[ d_{abs} = p_{<15}*f*n_{daf}+p_{falta}*f*n_{falta}\]

\subparagraph{Variável Intermediária
Acidentes}\label{variavel-intermediaria-acidentes}

Número de Acidentes

\[n_{acidentesk} = f * p_{acidentes} \ \forall k \in K\]

\subparagraph{Afastamentos}\label{afastamentos}

Afastamentos \textless{} 15 dias:
\[n_{afast<15k} = p_{<15} * n_{acidentes}  \ \forall k \in K\]

Afastamentos \textgreater{} 15 dias:
\[n_{afast>15k} = p_{>15} * n_{acidentes}  \ \forall k \in K\] Óbitos:
\[n_{obitosk} = p_{obitos} *n_{acidentes}  \ \forall k \in K \]

Acidentes Sem Afastamento:
\[n_{safastk} = p_{obitos} *n_{acidentes}  \ \forall k \in K \]

Dias de Absenteísmo (Antigo)

Os dias de absenteísmo \(d_{abs}\) serão estimados de acordo com a
probabilidade de absenteísmo \(\rho_{abs}\), número de funcionários
\(f\) e número de dias úteis do ano \(d\):

\[d_{abs} = \rho_{abs} * f * d\]

\subparagraph{Presenteísmo}\label{presenteismo}

\subparagraph{Produtividade}\label{produtividade}

\subparagraph{Qualidade}\label{qualidade}

\subparagraph{Refugo e Retrabalho}\label{refugo-e-retrabalho}

\subparagraph{MP, Insumos, Equipamentos
Operação}\label{mp-insumos-equipamentos-operacao}

\subsubsection{Variáveis Intermediárias}\label{variaveis-intermediarias}

\subsubsection{Custos}\label{custos}

\paragraph{Custos de Implementação}\label{custos-de-implementacao}


\end{document}
