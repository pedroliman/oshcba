\documentclass[]{article}
\usepackage{lmodern}
\usepackage{amssymb,amsmath}
\usepackage{ifxetex,ifluatex}
\usepackage{fixltx2e} % provides \textsubscript
\ifnum 0\ifxetex 1\fi\ifluatex 1\fi=0 % if pdftex
  \usepackage[T1]{fontenc}
  \usepackage[utf8]{inputenc}
\else % if luatex or xelatex
  \ifxetex
    \usepackage{mathspec}
  \else
    \usepackage{fontspec}
  \fi
  \defaultfontfeatures{Ligatures=TeX,Scale=MatchLowercase}
\fi
% use upquote if available, for straight quotes in verbatim environments
\IfFileExists{upquote.sty}{\usepackage{upquote}}{}
% use microtype if available
\IfFileExists{microtype.sty}{%
\usepackage{microtype}
\UseMicrotypeSet[protrusion]{basicmath} % disable protrusion for tt fonts
}{}
\usepackage[margin=1in]{geometry}
\usepackage{hyperref}
\hypersetup{unicode=true,
            pdftitle={Modelo Matematico CBR},
            pdfauthor={P.N. de Lima, D. B. Goldmeyer, L. F. R. Camargo, A. Dresch, D. P. Lacerda, T. Kunrath},
            pdfborder={0 0 0},
            breaklinks=true}
\urlstyle{same}  % don't use monospace font for urls
\usepackage{graphicx,grffile}
\makeatletter
\def\maxwidth{\ifdim\Gin@nat@width>\linewidth\linewidth\else\Gin@nat@width\fi}
\def\maxheight{\ifdim\Gin@nat@height>\textheight\textheight\else\Gin@nat@height\fi}
\makeatother
% Scale images if necessary, so that they will not overflow the page
% margins by default, and it is still possible to overwrite the defaults
% using explicit options in \includegraphics[width, height, ...]{}
\setkeys{Gin}{width=\maxwidth,height=\maxheight,keepaspectratio}
\IfFileExists{parskip.sty}{%
\usepackage{parskip}
}{% else
\setlength{\parindent}{0pt}
\setlength{\parskip}{6pt plus 2pt minus 1pt}
}
\setlength{\emergencystretch}{3em}  % prevent overfull lines
\providecommand{\tightlist}{%
  \setlength{\itemsep}{0pt}\setlength{\parskip}{0pt}}
\setcounter{secnumdepth}{5}
% Redefines (sub)paragraphs to behave more like sections
\ifx\paragraph\undefined\else
\let\oldparagraph\paragraph
\renewcommand{\paragraph}[1]{\oldparagraph{#1}\mbox{}}
\fi
\ifx\subparagraph\undefined\else
\let\oldsubparagraph\subparagraph
\renewcommand{\subparagraph}[1]{\oldsubparagraph{#1}\mbox{}}
\fi

%%% Use protect on footnotes to avoid problems with footnotes in titles
\let\rmarkdownfootnote\footnote%
\def\footnote{\protect\rmarkdownfootnote}

%%% Change title format to be more compact
\usepackage{titling}

% Create subtitle command for use in maketitle
\newcommand{\subtitle}[1]{
  \posttitle{
    \begin{center}\large#1\end{center}
    }
}

\setlength{\droptitle}{-2em}
  \title{Modelo Matematico CBR}
  \pretitle{\vspace{\droptitle}\centering\huge}
  \posttitle{\par}
  \author{P.N. de Lima, D. B. Goldmeyer, L. F. R. Camargo, A. Dresch, D. P.
Lacerda, T. Kunrath}
  \preauthor{\centering\large\emph}
  \postauthor{\par}
  \predate{\centering\large\emph}
  \postdate{\par}
  \date{junho de 2017}


\begin{document}
\maketitle

{
\setcounter{tocdepth}{6}
\tableofcontents
}
\section{Modelo Matemático - Razão
Benefício-Custo}\label{modelo-matematico---razao-beneficio-custo}

Este documento contém uma definição do modelo matemático que suporta a
calculadora de custos e benefícios de inciativas em SST.

\subsection{CBR - Razão
Benefício-Custo}\label{cbr---razao-beneficio-custo}

A razão benefício-custo \(\alpha\) corresponde à razão do somatório dos
custos \(C_i\) onde \(i\) representa o índice de custos e \(B_j\) os
benefícios a valor presente.

\[\alpha = \frac{\sum_{i=1}^{I} B_{i}} {\sum_{j=1}^{J} C_{j}}\]

\subsubsection{Fluxo de Caixa em Valor
Presente}\label{fluxo-de-caixa-em-valor-presente}

Os fluxos de caixa devem ser ajustados a valor presente utilizando-se
uma taxa de atratividade \(\theta\) definida pelo usuário do modelo. Tal
taxa será utilizada para trazer os valores de fluxo de caixa a valor
presente.

\[B_i(t) = \frac{b_i}{(1+\theta)^t}\]

\subsubsection{Calculo dos Benefícios}\label{calculo-dos-beneficios}

Em todos os casos, o benefício será calculado a partir da diferença em
valores monetários de uma variável financeira sem a iniciativa em SST e
com a iniciativa em SST. Exemplificando, o benefício gerado pela redução
de absenteísmo \(B_{abs}\) será calculado a partir da seguinte equação.

\[B_i = {D}_{i, ci} - {D}_{i, si}\] Exemplificando, se uma empresa, sem
uma iniciativa em SST terá \(20000\) reais em desepesas com absenteísmo,
e com esta iniciativa terá \(15000\), o benefício oriúndo desta
inciativa, apenas relacionado a absenteísmo será:

\[B_{abs} = {D}_{abs, ci} - {D}_{abs, si} = (-15000)-(-20000) = 5000\]

\paragraph{Despesas Evitáveis}\label{despesas-evitaveis}

\subparagraph{Despesas com Reclamatórias
Trabalhistas}\label{despesas-com-reclamatorias-trabalhistas}

\subparagraph{Acidente / Doença Ocupacional -
Invalidez}\label{acidente-doenca-ocupacional---invalidez}

\subparagraph{Ações Regressivas (NTEP)}\label{acoes-regressivas-ntep}

\subparagraph{Ausência para Tratamento}\label{ausencia-para-tratamento}

\subparagraph{Despesas Médicas}\label{despesas-medicas}

\subparagraph{Taxa de Sinistralidade}\label{taxa-de-sinistralidade}

\subparagraph{Interrupção Operacional
(Acidente/Morte)}\label{interrupcao-operacional-acidentemorte}

\subparagraph{Reabilitação do
trabalhador}\label{reabilitacao-do-trabalhador}

\paragraph{Benefício Não Capturado}\label{beneficio-nao-capturado}

\subparagraph{Exposição à Multas}\label{exposicao-a-multas}

\subparagraph{Reduções Fiscais}\label{reducoes-fiscais}

FAP, RAT e SAT: As despesas com seguro acidentário do trabalho
\(D_{sat}\) serão calculadas de acordo com as estimativas do \(FAP\)
(\(0,005 - 0,02\)) e \(RAT\) .

\[D_{sat} = RAT_{ajust}* F\]

Rat Ajustado

\[RAT_{ajust} = (FAP * RAT)\] O \(RAT\) varia entre 1 e 3, de acordo com
o cnae da empresa em questão. \[RAT \in {1,2,3}\]

FAP

O FAP, por sua vez, é calculado de acordo com os percentis de gravidade
\(p_g\), frequência \(p_{f}\) e custo\(p_c\):

\[FAP = (0,5*p_g + 0,35*p_{f}+0,15*p_c)0,02\]

Percentis são calculados de acordo com os índices nos dois anos
anteriores.* Os percentis dependem do posicionamento da empresa em
relação às demais. Específicamente a função \(Pos(I_{t-1},I_{t-2})\) é
calculada pela previdência de acordo com os índices de todas as empresas
no mesmo subgrupo do CNAE da empresa em questão. Ainda não foi definida
uma maneira de estimar esta função.

\[p_t = \frac{100*(Pos(I_{t-1},I_{t-2})-1)}{n-1}\]

Índice de Frequência

\[I_f = \frac{(n_{cats}+n_{b92}+n_{b91}+n_{b93})}{f} * 1000\]

Índice de gravidade

\[I_g = \frac{(0.1*n_{b91}+0.3*n_{b92}+0.5*n_{b93}+0.1*n_{b94})}{f}* 1000\]

Índice de Custo

\[I_c = \frac{c_{beneficios inss}}{folha media}/ * 1000\]

Custos de Benefícios: Considerar tempo médio de afastamento por tipo de
afastamento (b91, b92, b93, b94) e o ``ticket médio'' de cada um destes
benefícios.

\paragraph{Intangível}\label{intangivel}

\subparagraph{Imagem da Empresa}\label{imagem-da-empresa}

\subparagraph{Engajamento / Clima
Organizacional}\label{engajamento-clima-organizacional}

\paragraph{Melhor Uso dos Recursos}\label{melhor-uso-dos-recursos}

\subparagraph{Turnover}\label{turnover}

As despesas com Turnover \(D_{tur}\) serão calculadas com base no número
de funcionários afastados por problemas relacionados à SST \(n_{afast}\)
e no custo médio de substituição dos funcionários\(c_{sub}\).

\[D_{tur} = n_{afast} * c_{sub}\]

Número de Afastamentos é calculado de acordo com a probabilidade de
morte \(p_{morte}\) e a probabilidade de afastamento por período menor
que 15 dias \(p_{>15}\)

\[n_{afast} = p_{morte}*f + p_{>15}*f\]

Número de Afastamentos

O número de afastamentos \(n_{afast}\) será estimado de acordo com a
probabilidade de afastamento \(\rho_{afast}\) e número de funcionários
\(f\): \[n_{afast} = \rho_{afast} * f \]

\subparagraph{Absenteísmo}\label{absenteismo}

As despesas com Absenteísmo \(D_{abs}\) serão calculadas com base no
número de dias de absenteísmo por problemas relacionados à SST
\(d_{abs}\), no número de horas trabalhadas por dia \(h\) e no custo em
mão de obra médio horário \(c_{mdo}\).

\[D_{abs} = d_{abs} * h * c_{mdo}\]

Dias de Absenteísmo

\[ d_{abs} = p_{<15}*f*n_{daf}+p_{falta}*f*n_{falta}\]

Dias de Absenteísmo (Antigo)

Os dias de absenteísmo \(d_{abs}\) serão estimados de acordo com a
probabilidade de absenteísmo \(\rho_{abs}\), número de funcionários
\(f\) e número de dias úteis do ano \(d\):

\[d_{abs} = \rho_{abs} * f * d\]

\subparagraph{Presenteísmo}\label{presenteismo}

\subparagraph{Produtividade}\label{produtividade}

\subparagraph{Qualidade}\label{qualidade}

\subparagraph{Refugo e Retrabalho}\label{refugo-e-retrabalho}

\subparagraph{MP, Insumos, Equipamentos
Operação}\label{mp-insumos-equipamentos-operacao}

\subsubsection{Variáveis Intermediárias}\label{variaveis-intermediarias}

\subsubsection{Custos}\label{custos}

\paragraph{Custos de Implementação}\label{custos-de-implementacao}


\end{document}
